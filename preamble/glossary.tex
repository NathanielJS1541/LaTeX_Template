% Generate, define and style the glossaries for this document. This must be used
% as an \input in the preamble.

% Store the \gls, \glspl, and \glsxtrshortpl as "old" commands, so they can be
% extended or modified.
\let\oldgls\gls
\let\oldglspl\glspl
\let\oldglsxtrshortpl\glsxtrshortpl

% Redefine \gls, \glspl, and \glsxtrshortpl to use custom hyperlink formatting
% (bold and underlined). This is more readable and easier to follow when the
% document is printed out.
\renewcommand{\gls}[1]{\textbf{\uline{\oldgls{#1}}}}
\renewcommand{\glspl}[1]{\textbf{\uline{\oldglspl{#1}}}}
\renewcommand{\glsxtrshortpl}[1]{\textbf{\uline{\oldglsxtrshortpl{#1}}}}

% Customise how the "see" and "seealso" entries are rendered by renewing the
% \glsseeformat command.
\renewcommand\glsseeformat[3][\seename]
{
    % Non-breaking new line, so the references appear on the line below the
    % glossary entry.
    \\*
    % Emphasize the first parameter (default is \seename), and display the list
    % of "see" entries as bold and underlined to be consistent with the \gls,
    % \glspl, \glsxtrshortpl commands.
    \emph{#1} \textbf{\uline{\glsseelist{#2}}}
}

% Generate the glossaries for this document.
\makeglossaries

% Define the terms for the glossary.
% Example entries:
\newglossaryentry{box}
{
    name=box,
    % Specify the plural if it requires more than appending an s...
    plural=boxes,
    description={A container with a flat base and sides, typically square or
    rectangular, and often having a lid.}
}
\newglossaryentry{cardboard box}
{
    name=cardboard box,
    % Specify the plural if it requires more than appending an s...
    plural=cardboard boxes,
    description={A \gls{box} made out of cardboard.},
    seealso={box}
}

% Define the acronyms for the glossary.
% Example entries:
\newacronym{pdf}{PDF}{Portable Document Format.}
\newacronym{png}{PNG}{Portable Network Graphics.}
\newacronym{curl}{curl}{\Gls{curl} URL Request Library.}

% Set the style of the glossary when added to a page.
% Here is a list of common styles:
% - 'list': A simple list of glossary entries without any grouping.
% - 'index': Similar to 'list', but entries are sorted like an index.
% - 'altlist': An alternative list style with a different layout.
% - 'tree': Displays entries in a tree-like structure.
% - 'super': A super-tabular style for multi-column glossary entries.
% - 'long': Uses the longtable environment for multi-page glossaries.
% - 'listhypergroup': Style groups entries by their first letter and hyperlinks
%   each group.
% - 'altlisthypergroup': Similar to 'listhypergroup' but with an alternative
%   layout.
\setglossarystyle{listhypergroup}